\section{Progettazione Sottoreti}
Per far si che in una rete avvenga lo scambio di pacchetti è necessario che gli host, coloro che inviano e ricevono dati, siano identificati tramite un indirizzo dedicato (indirizzo IP) che lo rappresenta e lo distingue dagli altri.
Dunque ogni host della rete necessita del suo indirizzo IP per comunicare gli altri host della rete e con quelli delle altre reti.

\subsection{Indirizzi IP pubblici e privati}

Bisogna distinguere gli indirizzi usati nella LAN (indirizzi privati) con quelli usati quando si comunica con un host di un altra rete (indirizzi pubblici).

Gli \textbf{Indirizzi privati} sono utilizzati per indentificare gli host all'interno di una stessa rete e vengono distinti in 3 categorie:

\begin{tabular}{|c|c|c|c|}
    \hline
    \thead{Categoria} &\thead{Indirizzo Partenza} &\thead{Indirizzo Fine} &\thead{Host Disponibili}\\
    \hline
    Classe A & 10.0.0.0 & 10.255.255.255 & 16'777'216\\
    \hline
    Classe B & 172.16.0.0 & 172.31.255.255 & 1'048'576\\
    \hline
    Classe C & 192.168.0.0 & 192.268.255.255 & 65'536\\
    \hline
\end{tabular}


Gli \textbf{Indirizzi pubblici} sono utilizzati per identificare gli host fuori dalla LAN, ovvero essere riconosciuto dagli host delle altre reti con la quale sta avvenendo una comunicazione. Fanno parte di questa categoria tutti gli indirizzi che non appartengono alle classi precedentemente elencate.

Per assegnare ai dispositivi della rete gli indirizzi IP esistono due modi:
\begin{itemize}
    \item Automatico: Grazie al DHCP, un servizio che consente di assegnare gli indirizzi IP in modo automatico, effettuato o tramite router o tramite server.
    \item Manuale: Inserimento degli indirizzi manuale per ogni host
\end{itemize}

In questa guida vedremo come configurare una LAN, con i relativi servizi e il calcolo degli indirizzi IP per ogni sottorete con il subnetting VLSM.

\subsection{Subnetting VLSM}
La prima cosa da fare in una nuova rete è determinare gli indirizzi IP privati per le sottoreti che sono presenti, passaggio obbligatorio sia se si vuole assegnare gli indirizzi con DHCP o manualmente.

Il subnettig VLSM è una tecnica che consente di calcolare gli indirizzi IP necessari per ogni sottorete senza "sprechi", ossia evitare di riservare una quantita di indirizzi IP superiore alla reale necessità.
Proprio per evitare questo spreco di indirizzi nel VLSM è necessario iniziare sempre dalla sottorete con il numero di host maggiore.

Nel nostro caso prendiamo la sottorete di partenza che necessita di indirizzi IP per 300 host, e gli assegnamo come indirizzo di partenza 192.168.6.0 (che come visto prima fa parte della classe C degli indirizzi privati). Successivamente calcoliamo il numero di bit necessari per gli host richiesti, esso equivale all’esponente della potenza del due che come risultato deve raggiungere almeno il numero di host (300 in questo caso) + 2. 
\newline
Noi useremo 9 bit, in quanto $2^9$ = 512 e a cui sottraendo il broadcast e l’indirizzo di partenza, otteniamo 510 host disponibili, che superano di gran lunga i 300 di cui necessitavamo. In questo caso lo spreco di indirizzi non avviene in quando se avessimo usato 8 bit, il numero risultante non sarebbe stato sufficiente a soddisfare la richiesta di 300 host.
A questo punto inizia il vero calcolo degli indirizzi. Iniziamo a calcolare la subnet mask (255.255.255.255 in numero decimale), e la riscriviamo in binario, che vanno a formare quattro ottetti tutti da 1 (11111111.11111111.11111111.11111111). Una volta fatto ciò, partendo da destra, sostituiamo con gli zeri il numero di bit pari al numero di bit precedentemente calcolato. In questo caso essendo 9 il risultato sarà 11111111.11111111.11111110.00000000 che una volta trasformato in decimale rivelerà la subnet mask della sottorete per 300 host (255.255.254.0). 
Come ultimo passaggio bisogna calcolare il broadcast, ossia l'ultimo indirizzo disponibile per la sottorete. Iniziamo convertendo in binario il nostro indirizzo di partenza (192.168.6.0 - 11000000.10101000.00000110.00000000), e sempre partendo da destra, contiamo il numero di bit rich iesti al inizio (cioè 9) e sostituiamo quest'ultimi con gli uno. Quindi in questo caso il risultato sarà 11000000.10101000.00000111.11111111, che una volta trasformato in numero decimale equivarrà al nostro indirizzo broadcast (192.168.7.255).
A questo punto abbiamo ricavato il range della prima sottorete. Continuando il calcolo degli indirizzi per la seconda sottorete che conta il numero maggiore di host, bisognerà impostare come indirizzo di partenza l'indirizzo di broadcast + 1 della sottorete precedentemente calcolata (in questo caso 192.168.8.0) e continuare come già descritto.

\subsection{Tabella Sottoreti}
% 
Per ogni rete che viene progettata sarebbe ideale compilare una tabella delle sottoreti (come quella riportata qui sotto) nella quale andrebbero inseriti tutti gli indirizzi e informazioni riguardanti ogni singola sottorete in modo da avere un controllo e un'organizzazione migliore sull'intera rete.
\newline
Qui sotto viene riportata la tabella delle sottoreti ricavata dal calcolo degli indirizzi IP tramite subnetting VLSM.
\newline
\begin{tabular}{|c|c|c|c|c|c|c|c|c|} 
    \hline
    \thead{Nome} & \thead{Indirizzo} & \thead{Subnet\\mask} & \thead{Broadcast} & \thead{Indirizzi\\Disp.} & \thead{Gateway} & \thead{Router} & \thead{Interfaccia}\\
    \hline
    \Centerstack{Host\\300} & 192.168.6.0 & 255.255.254.0 & 192.168.7.255 & 510 & 192.168.6.1 & VPN-SX & Gig0/0\\
    \hline
    \Centerstack{Host\\100} & 192.168.8.0 & 255.255.255.128 & 192.168.8.127 & 126 & 192.168.8.1 & Router 1 & Fa0/0\\
    \hline
    \Centerstack{Host\\60} & 192.169.8.128 & 255.255.255.192 & 192.168.8.191 & 62 & 192.168.8.129 & VPN-DX & Gig0/1\\
    \hline
    \Centerstack{Host\\10} & 192.168.8.192 & 255.255.255.240 & 192.168.8.207 & 14 & 192.168.8.193 & Router 0 & Fa1/0\\
    \hline
    \Centerstack{Host\\2} & 192.168.8.208 & 255.255.255.252 & 192.168.8.211 &  2 &  & VPN-DX & Gig0/1\\
    \hline
    \Centerstack{Host\\2 Dx} & 192.168.8.212 & 255.255.255.252 & 192.168.8.215 &  2 & & VPN-DX & \Centerstack{Gig0/1\\Fa1/0}\\
    \hline
    \Centerstack{Host\\2 Sx} & 192.168.8.216 & 255.255.255.252 & 192.168.8.219 &  2 & & VPN-DX & \Centerstack{Gig0/1\\Fa1/0}\\
    \hline
    \Centerstack{Tunnel\\1 Sx} & 192.168.9.0 & 255.255.255.0 & 192.168.9.255 &  255 & & \Centerstack{Tunnel 1\\Router 0} & \Centerstack{Gig0/0\\Fa6/0}\\
    \hline
    \Centerstack{Tunnel\\2 Dx} & 192.168.10.0 & 255.255.255.0 & 192.168.10.255 &  255 & & \Centerstack{Tunnel 2\\Router 1} & \Centerstack{Gig0/0\\Fa6/0}\\
    \hline
    \Centerstack{IPv6\\Dx} & 1000:: & /64 & & & 1000::1 & Tunnel 2 & Gig0/1\\
    \hline
    \Centerstack{IPv6\\Sx} & 3000:: & /64 & & & 3000::1 & Tunnel 1 & Gig0/1\\
    \hline

\end{tabular}
