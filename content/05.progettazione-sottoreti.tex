\section{Progettazione Sottoreti}
Per far si che in una rete avvenga lo scambio di pacchetti è necessario che gli host, coloro che inviano e ricevono dati, siano identificati tramite un indirizzo dedicato (indirizzo IP) che lo rappresenta e lo distingue dagli altri.
Dunque ogni host della rete necessita del suo indirizzo IP per comunicare gli altri host della rete e con quelli delle altre reti.

\subsection{Indirizzi IP pubblici e privati}

Bisogna distinguere gli indirizzi usati nella LAN (indirizzi privati) con quelli usati quando si comunica con un host di un altra rete (indirizzi pubblici).

Gli \textbf{indirizzi privati} sono utilizzati per indentificare gli host all'interno di una stessa rete e vengono distinti in 3 classi:

\medskip

\begin{tabular}{cccc}
     & \textbf{Indirizzo partenza} & \textbf{Indirizzo fine} & \textbf{Totale}\\
    \textbf{Classe A} & 10.0.0.0 & 10.255.255.255 & 16,777,216\\
    \textbf{Classe B} & 172.16.0.0 & 172.31.255.255 & 1,048,576\\
    \textbf{Classe C} & 192.168.0.0 & 192.268.255.255 & 65,536\\
\end{tabular}

\medskip

Gli \textbf{indirizzi pubblici} sono utilizzati per identificare gli host fuori dalla LAN, ovvero essere riconosciuto dagli host delle altre reti con la quale sta avvenendo una comunicazione. Fanno parte di questa categoria tutti gli indirizzi che non appartengono alle classi precedentemente elencate.

Per assegnare ai dispositivi della rete gli indirizzi IP esistono due modi:
\begin{itemize}
    \item Automatico: Grazie al DHCP, un servizio che consente di assegnare gli indirizzi IP in modo automatico, effettuato o tramite router o tramite server.
    \item Manuale: Inserimento degli indirizzi manuale per ogni host
\end{itemize}

In questa guida vedremo come configurare una LAN, con i relativi servizi e il calcolo degli indirizzi IP per ogni sottorete con il subnetting VLSM.

\subsection{Subnetting VLSM}
La prima cosa da fare in una nuova rete è determinare gli indirizzi IP privati per le sottoreti che sono presenti, passaggio obbligatorio sia se si vuole assegnare gli indirizzi con DHCP o manualmente.

Il subnetting VLSM è una tecnica che consente di calcolare gli indirizzi IP necessari per ogni sottorete senza ``sprechi'', ossia evitare di riservare una quantita di indirizzi IP superiore alla reale necessità.
Proprio per evitare questo spreco di indirizzi nel VLSM è necessario iniziare sempre dalla sottorete con il numero di host maggiore.

Nel nostro caso prendiamo la sottorete di partenza che necessita di indirizzi IP per 300 host, e gli assegnamo come indirizzo di partenza 192.168.6.0. Successivamente calcoliamo il numero di bit necessari per gli host richiesti, esso equivale all’esponente della potenza del due che ha come risultato un numero uguale o superiore al numero di host più due, formalmente, indicando con $n$ il numero di host richiesti, il numero di bit dedicati all'identificazione dell'host equivale a $\lceil \log_2 (n + 2) \rceil$

Quindi nel caso ipotetico di 128 host, nonostante $2^7 = 128$, saremmo costretti a utilizzare 8 bit siccome il primo indirizzo (quello con solo zeri nella parte degli host) e l'ultimo indirizzo (quello con soli uno nella parte degli host) sono indirizzi speciali il primo dedicato a identificare la rete e il secondo dedicato alla comunicazione broadcast.

Nel nostro caso le potenze del due più vicine a $300$ sono $2^8 = 256$ e $2^9 = 512$ quindi sono necessari 9 bit per ospitare tutti i $300$ host richiesti.

A questo punto inizia il vero calcolo degli indirizzi. Iniziamo a calcolare la subnet mask partendo da un'indirizzo IP formato esclusivamente da uno:

\begin{center}
    \begin{tabular}{c@{.}c@{.}c@{.}c}
        255 & 255 & 255 & 255\\
        11111111 & 11111111 & 11111111 & 11111111
    \end{tabular}
\end{center}

Una volta fatto ciò, sostituiamo la parte dell'indirizzo IP dedicata agli host (nel nostro caso 9 bit partendo da destra) con gli zeri

\begin{center}
    \begin{tabular}{c@{.}c@{.}c@{.}c}
        255 & 255 & 254 & 0\\
        11111111 & 11111111 & 1111111\textcolor{red}{0} & \textcolor{red}{00000000}
    \end{tabular}
\end{center}

Come ultimo passaggio bisogna calcolare il broadcast, ossia l'ultimo indirizzo disponibile per la sottorete. Iniziamo convertendo in binario il nostro indirizzo di rete

\begin{center}
    \begin{tabular}{c@{.}c@{.}c@{.}c}
        192 & 168 & 6 & 0\\
        11000000 & 10101000 & 00000110 & 00000000
    \end{tabular}
\end{center}

\noindent e come prima sostituiamo la parte degli host ma questa volta con degli uno

\begin{center}
    \begin{tabular}{c@{.}c@{.}c@{.}c}
        192 & 168 & 7 & 255\\
        11000000 & 10101000 & 0000011\textcolor{red}{1} & \textcolor{red}{11111111}
    \end{tabular}
\end{center}

A questo punto abbiamo ricavato il range della prima sottorete. Continuando il calcolo degli indirizzi per la seconda sottorete che conta il numero maggiore di host, bisognerà impostare come indirizzo di partenza l'indirizzo di broadcast + 1 della sottorete precedentemente calcolata (in questo caso 192.168.8.0) e continuare come già descritto.

\subsection{Tabella Sottoreti}
Per ogni rete che viene progettata sarebbe ideale compilare una tabella delle sottoreti (come quella riportata qui sotto) nella quale andrebbero inseriti tutti gli indirizzi e informazioni riguardanti ogni singola sottorete in modo da avere un controllo e un'organizzazione migliore sull'intera rete.

Qui sotto viene riportata la tabella delle sottoreti ricavata dal calcolo degli indirizzi IP tramite subnetting VLSM.

\begin{adjustbox}{center}
    \begin{tabular}{rccccccc}
        \toprule
        \multicolumn{1}{c}{Nome} & Network & Subnet mask & Broadcast & \shortstack{Indirizzi\\Disponibili} & Gateway & Router & Interfaccia\\
        \midrule 
        % Nome      & Network       & Subnet          & Broadcast      & HT  & Gateway       & Router               & Interfaccia
        Host 300    & 192.168.6.0   & 255.255.254.0   & 192.168.7.255  & 510 & 192.168.6.1   & VPN-SX               & Gig0/0\\
        Host 100    & 192.168.8.0   & 255.255.255.128 & 192.168.8.127  & 126 & 192.168.8.1   & Router 1             & Fa0/0\\
        Host 60     & 192.169.8.128 & 255.255.255.192 & 192.168.8.191  & 62  & 192.169.8.129 & VPN-DX               & Gig0/1\\
        Host 10     & 192.168.8.192 & 255.255.255.240 & 192.168.8.207  & 14  & 192.168.8.193 & Router 0             & Fa1/0\\
        Host 2      & 192.168.8.208 & 255.255.255.252 & 192.168.8.211  & 2   &               & VPN-DX               & Gig0/1\\
        Host 2 Dx   & 192.168.8.212 & 255.255.255.252 & 192.168.8.215  & 2   &               & VPN-DX               & Gig0/1 \& Fa1/0\\
        Host 2 Sx   & 192.168.8.216 & 255.255.255.252 & 192.168.8.219  & 2   &               & VPN-DX               & Gig0/1 \& Fa1/0\\
        Tunnel 1 Sx & 192.168.9.0   & 255.255.255.0   & 192.168.9.255  & 255 &               & Tunnel 1 \& Router 0 & Gig0/0 \& Fa6/0\\
        Tunnel 2 Dx & 192.168.10.0  & 255.255.255.0   & 192.168.10.255 & 255 &               & Tunnel 2 \& Router 1 & Gig0/0 \& Fa6/0\\
        IPv6 Dx     & 1000::        & /64             &                &     & 1000::1       & Tunnel 2             & Gig0/1\\
        IPv6 Sx     & 3000::        & /64             &                &     & 3000::1       & Tunnel 1             & Gig0/1\\
        \bottomrule
    \end{tabular}%
\end{adjustbox}